\documentclass[a4paper]{article}

\usepackage{verbatim}
\usepackage{fancyhdr}
\usepackage{latexsym}
\usepackage[colorlinks=true,
	linkcolor=blue,
	anchorcolor=black,
	citecolor=black,
	filecolor=black,
	menucolor=black,
	pagecolor=black,
	urlcolor=black
	]{hyperref}
\usepackage{tabularx}
\usepackage[left=1in,top=1in,right=1in]{geometry}

\pagestyle{fancy}
\fancyhf{}
\fancyhf[FR]{Page \thepage}
%\addtolength{\headheight}{3pt}
%\addtolength{\textheight}{48pt}
\renewcommand{\headrulewidth}{1pt}
\renewcommand{\footrulewidth}{1pt}
\fancypagestyle{first}{%
	\fancyhf[HR]{}
	}

\newcommand{\head}[1]{%
	\vspace{18pt}
	\large {\underline{#1}}
	\normalsize
	\vspace{12pt}
	}

\renewcommand{\author}{\small{\\by\\Satya\\https://www.github.com/satyap}}

\setlength{\parindent}{0pt}
\setlength{\parskip}{6pt}

%% the following newcommands are mostly for the math sheets
\newcommand{\elemof}[1]{%
    \in \mathbf{#1}
}
\newcommand{\degree}{^\circ}
\newcommand{\h}[2]{#2\footnote{\href{#1}{#1}} }
\newcommand{\p}{\vspace{12pt}}
\newcommand{\cosec}{\mathrm{cosec}}
\newcommand{\ud}{\mathrm{d}}
\newcommand{\half}{\frac{1}{2}}

%%%%%%%%%%%%%%%%%%%%%


\fancyhf[FL]{Signals \& Systems Formul\ae by Satya}

%%%%%%%%%%%%%%%%%%%%%

\begin{document}

\thispagestyle{first}

\begin{center}
\LARGE { \textbf{Signals \& Systems formul\ae} }
\author
\end{center}

\tableofcontents

\section{General}

Energy/Power Signal
\begin{displaymath}
P=
\lim_{T \rightarrow 0}
\frac {1} {2\pi}
\int_{-T}^{T}
|g(t)^2|
dt
\textrm{ = average power by $g(t)$ over time $T$ ($2T$?)}
\end{displaymath}


\begin{displaymath}
E=
\int_{-\infty}^{\infty}
|g(t)^2
dt
\textrm{ = total energy in g(t)}
\end{displaymath}


Functions $f(t)$ and $g(t)$ are orthogonal in interval $a$ to $b$ if
\begin{displaymath}
\int_a^b f(t) g(t) dt =0
\end{displaymath}


\section{Fourier Series}

\begin{displaymath}
x(t)= a_0 +2
\sum_{n=1}^\infty 
(a_n \cos {n\omega_0 t} + b_n \sin n\omega_0 t)
\end{displaymath}

\begin{displaymath}
\omega_0=\frac{2\pi}{T_0}
\qquad
a_0=\frac{1}{T_0}
\int_0^{T_0} x(t) dt
\end{displaymath}

\begin{displaymath}
a_n = \frac{1}{T_0}
\int_0^{T_0} x(t) \cos n \omega_0 t dt
\quad 
b_n = \frac{1}{T_0}
\int_0^{T_0} x(t) \sin n \omega_0 t dt
\end{displaymath}

Polar form:
\begin{displaymath}
x(t)= c_0 +
\sum_{n=1}^\infty C_n \cos (n \omega t + \phi_n)
\end{displaymath}

\begin{displaymath}
c_0=a_0
\qquad
C_n=\sqrt{a_n^2 + b_n^2}
\qquad
\phi_n=-\tan^{-1} \frac{b_n}{a_n}
\end{displaymath}

Exponential form:

\begin{displaymath}
x(t)=
\sum_{n=-\infty}^\infty C_n e^{jn\omega_0 t}
\qquad
C_n=
\frac{1}{T_0}
\int_0^{T_0} x(t) e^{-j\omega_0 nt} dt
\end{displaymath}


Spectrum of trignometric series exists for positive $\omega$ only.

For exponential form, for real $x(t)$, magnitude spectrum is even, phase
spectrum is odd.

Magnitude spectrum -ve sign $\Rightarrow180^\circ$ phase shift.

dc values= $a_0$ $S_0$ $C_0$ etc.

Even function: $b_n=0$,  Odd function: $a_0=a_n=0$

\begin{tabular}{l l}
$S_n$ v/s $\omega$ :    & Amplitude spectrum\\
$\phi_n$ v/s $\omega$ : & Phase spectrum\\
\end{tabular}


\begin{displaymath}
a_0=c_0
\qquad
a_n=(C_n + C_{-n})
\qquad
b_n=j(C_n - C_{-n})
\end{displaymath}

\begin{displaymath}
C_n=\frac{1}{2}(a_n - jb_n)
\qquad
C_{-n}=\frac{1}{2}(a_n + jb_n)
\end{displaymath}


\section{Fourier and Inverse Fourier Transforms}

\begin{displaymath}
F(\omega)=\int_{-\infty}^{\infty} f(t) e^{-j\omega t} dt
\qquad
f(t)=\frac{1}{2\pi} \int_{-\infty}^{\infty} F(\omega) e^{j\omega t}d\omega
\end{displaymath}

Fourier transform exists for $f(t)$ if:

\begin{displaymath}
\textrm{Energy } E = \int_{-\infty}^{\infty} |f(t)|^2 dt < \infty
\end{displaymath}

Phase spectrum does not exist if $F(\omega)$ is
real. Magnitude $M=|F(\omega)|$

\begin{displaymath}
\pi(\frac{t}{\tau})=
1 \textrm{ for }
-\frac{\tau}{2}
< t <
\frac{\tau}{2}
\textrm{ and 0 elsewhere}
\end{displaymath}

\begin{displaymath}
sinc(x)=\frac{\sin \pi x}{\pi x}
\end{displaymath}

\begin{displaymath}
sgn(t)=1 \textrm{ for } t>0 \textrm{ and } -1 \textrm{ for } t<0
\end{displaymath}

Sampling property of unit impulse function:
\begin{displaymath}
\int_{-\infty}^{\infty} m(t) \delta (t-\tau) dt = m(\tau)
\end{displaymath}

\subsection{Properties of Fourier Transform}

\begin{displaymath}
\textrm{Linearity:  }
ax(t) + by(t)
\stackrel{F}{\longleftrightarrow}
aX(\omega) + bY(\omega)
\end{displaymath}

\begin{displaymath}
\textrm{Time shifting:  }
x(t-t_0)
\stackrel{F}{\longleftrightarrow}
e^{-j\omega t_0}X(\omega)
\end{displaymath}

\begin{displaymath}
\textrm{Frequency shifting:  }
x(t) e^{j\omega t_0}
\stackrel{F}{\longleftrightarrow}
X(\omega-\omega_0)
\end{displaymath}

\begin{displaymath}
\textrm{Differentiation:  }
\frac{d^nx(t)}{dt^n}
\stackrel{F}{\longleftrightarrow}
(j\omega)^nX(\omega)
\end{displaymath}

\begin{displaymath}
\textrm{Integration:  }
\int_{-\infty}^t x(t)dt
\stackrel{F}{\longleftrightarrow}
\frac{1}{j\omega}X(\omega) + \pi X(0)\delta(\omega)
\end{displaymath}

\begin{displaymath}
\textrm{Scaling:  }
x(at)
\stackrel{F}{\longleftrightarrow}
\frac{1}{|a|} X(\frac{\omega}{a})
\end{displaymath}

\begin{displaymath}
\textrm{Duality:  }
X(t)
\stackrel{F}{\longleftrightarrow}
2\pi x(-\omega)
\end{displaymath}

\begin{displaymath}
\textrm{Convolution:  }
x(t)*y(t)
\stackrel{F}{\longleftrightarrow}
X(\omega)Y(\omega)
\end{displaymath}

\begin{displaymath}
F^{-1} (\frac{1}{a+j\omega}) = e^{-at}u(t)
\end{displaymath}

\begin{displaymath}
\textrm{Parseval's theorem:  }
\int_{-\infty}^{\infty}
|x(t)|^2 dt = \frac{1}{2\pi}
\int_{-\infty}^{\infty}
|X(\omega)|^2 d\omega
\end{displaymath}

\begin{displaymath}
|X(\omega)|^2 = \textrm{ energy density spectrum of x(t)}
\end{displaymath}


\section{z-Transform}

\begin{displaymath}
X(z)=\sum_{n=-\infty}^{\infty} x(n)z^{-n}
\end{displaymath}

\begin{displaymath}
\textrm{If $z=re^{j\omega}$ then }
X(z)=z\{x(n)\}=F\{x(n)r^{-n}\}
\end{displaymath}

If $r=1$ then z transform reduces to Fourier transform. If ROC (Region of
Convergence) of ZT includes unit circle then $x(n)$ is FTable.

\begin{displaymath}
z\{a^nu(n)\}=\frac{z}{z-a} \textrm{ ROC: } z>a
\textrm{ Note special case when a=1}
\end{displaymath}

\begin{displaymath}
z\{-a^nu(-n-1)\}=\frac{z}{z-a} \textrm{ ROC: } z<a
\end{displaymath}

\subsection{Properties of z-Transform}

%%%%%%%%% ztrans

\begin{displaymath}
\textrm{Linearity:  }
ax(n) + by(n)
\stackrel{z}{\longleftrightarrow}
aX(z) + bY(z)
\textrm{  ROC: Intersection}
\end{displaymath}

\begin{displaymath}
\textrm{Time scaling:  }
a_nx(n)
\stackrel{z}{\longleftrightarrow}
X(a^{-1}z)
\qquad
r_1<z<r_2\longleftrightarrow ar_1<z<ar_2
\end{displaymath}

\begin{displaymath}
\textrm{Time shifting: }
x(n-k)
\stackrel{z}{\longleftrightarrow}
z^{-k}X(z)
\end{displaymath}
For $\infty$ duration series ROC remains same

\begin{displaymath}
\textrm{Time reversal: }
x(-n)
\stackrel{z}{\longleftrightarrow}
X(z^{-1})
\qquad
r_1<z<r_2\leftrightarrow \frac{1}{r_1}>z>\frac{1}{r_2}
\end{displaymath}

\begin{displaymath}
\textrm{Multiplication/differentiation: }
nx(n)
\stackrel{z}{\longleftrightarrow}
-zX'(z)
\end{displaymath}

\begin{displaymath}
\textrm{Division/integration: }
\frac{x(n)}{n}
\stackrel{z}{\longleftrightarrow}
-\int_0^z \frac{X(z)}{z}dz
\end{displaymath}

\begin{displaymath}
\textrm{Initial value thm: If $x(n)$ is causal then }
x(0)=\lim_{z\rightarrow \infty} X(z)
\end{displaymath}


\begin{displaymath}
\textrm{Convolution: }
x_1(n) * x_2(n)
\stackrel{z}{\longleftrightarrow}
X_1(z)\cdot X_2(z)
\end{displaymath}

\begin{displaymath}
x(n)=\sum_{k=-\infty}^{\infty} x_1(k)x_2(n-k)
\end{displaymath}

\begin{displaymath}
x(n-k) \stackrel{z}{\longleftrightarrow} 
z^{-k}[X^+(z) + 
\sum_{n=1}^{k} x(-n)z^n ]
\qquad k>0
\end{displaymath}

ROC $|z|>a_{max} \Rightarrow$ causal system

ROC $|z|<a_{min} \Rightarrow$ anticausal system

ROC includes unit circle $\Rightarrow$ stable system


\section{Laplace Transform}

\begin{displaymath}
L\{f(t)\}=F(s)=
\int_{-\infty}^{\infty} e^{-st}f(t)dt
\end{displaymath}

\begin{displaymath}
L\{f(t)\}=F\{f(t)e^{-\sum t}\}
\end{displaymath}

If ROC of LT includes $\sigma=0$ then f(t) is FTable.

Derivative:
\begin{displaymath}
\frac{d}{dt}f(t) 
\stackrel{L}{\longleftrightarrow}
aF(s)-f(0)
\end{displaymath}

\begin{displaymath}
\frac{d^2}{dt^2}f(t) 
\stackrel{L}{\longleftrightarrow}
a^2F(s)-sf(0)-f'(0)
\end{displaymath}

\begin{displaymath}
\frac{d^3}{dt^3}f(t) 
\stackrel{L}{\longleftrightarrow}
a^3F(s)-s^2f(0)-sf'(0)-f''(0)
\end{displaymath}


\begin{displaymath}
\textrm{Integral: }
\int f(t)dt
\stackrel{L}{\longleftrightarrow}
\frac{1}{s}F(s) + \frac{1}{s}f^{-1}(0)
\end{displaymath}

\begin{displaymath}
\textrm{Time shifting: }
f(t-a)
\stackrel{L}{\longleftrightarrow}
e^{-as}F(s)
\end{displaymath}


\begin{displaymath}
\textrm{Unit impulse: }
\delta(t)
\stackrel{L}{\longleftrightarrow}
1
\end{displaymath}

\begin{displaymath}
\textrm{Unit step: }
u(t)
\stackrel{L}{\longleftrightarrow}
\frac{1}{s}
\end{displaymath}

\begin{displaymath}
t^n
\stackrel{L}{\longleftrightarrow}
\frac{n!}{s+{n+1}}
\end{displaymath}

\begin{displaymath}
\sin \omega t
\stackrel{L}{\longleftrightarrow}
\frac{\omega}{s^2 + \omega ^2}
\end{displaymath}

\begin{displaymath}
\cos \omega t
\stackrel{L}{\longleftrightarrow}
\frac{s}{s^2 + \omega ^2}
\end{displaymath}

\begin{displaymath}
e^{-at}
\stackrel{L}{\longleftrightarrow}
\frac{1}{s+a}
\end{displaymath}

\begin{displaymath}
\textrm{Initial value: }
\lim_{t\rightarrow 0} f(t)
\stackrel{L}{\longleftrightarrow}
\lim_{s\rightarrow\infty} sF(s)
\end{displaymath}

\begin{displaymath}
\textrm{Final value: }
\lim_{t\rightarrow\infty} f(t)
\stackrel{L}{\longleftrightarrow}
\lim_{s\rightarrow 0} sF(s)
\end{displaymath}


\end{document}
