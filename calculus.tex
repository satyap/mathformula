\documentclass[a4paper]{article}

\usepackage{verbatim}
\usepackage{fancyhdr}
\usepackage{latexsym}
\usepackage[colorlinks=true,
	linkcolor=blue,
	anchorcolor=black,
	citecolor=black,
	filecolor=black,
	menucolor=black,
	pagecolor=black,
	urlcolor=black
	]{hyperref}
\usepackage{tabularx}
\usepackage[left=1in,top=1in,right=1in]{geometry}

\pagestyle{fancy}
\fancyhf{}
\fancyhf[FR]{Page \thepage}
%\addtolength{\headheight}{3pt}
%\addtolength{\textheight}{48pt}
\renewcommand{\headrulewidth}{1pt}
\renewcommand{\footrulewidth}{1pt}
\fancypagestyle{first}{%
	\fancyhf[HR]{}
	}

\newcommand{\head}[1]{%
	\vspace{18pt}
	\large {\underline{#1}}
	\normalsize
	\vspace{12pt}
	}

\renewcommand{\author}{\small{\\by\\Satya\\https://www.github.com/satyap}}

\setlength{\parindent}{0pt}
\setlength{\parskip}{6pt}

%% the following newcommands are mostly for the math sheets
\newcommand{\elemof}[1]{%
    \in \mathbf{#1}
}
\newcommand{\degree}{^\circ}
\newcommand{\h}[2]{#2\footnote{\href{#1}{#1}} }
\newcommand{\p}{\vspace{12pt}}
\newcommand{\cosec}{\mathrm{cosec}}
\newcommand{\ud}{\mathrm{d}}
\newcommand{\half}{\frac{1}{2}}

%%%%%%%%%%%%%%%%%%%%%


\newcommand{\derivx}[2]{%
\begin{array}{c}\underline{\ud #1} \\ \ud #2\end{array}
}
\newcommand{\derix}[1]{%
\begin{array}{c}\underline{\ud #1} \\ \ud x\end{array}
}
\newcommand{\derx}{%
\begin{array}{c}\ud  \\ \overline{\ud x}\end{array}
}

\fancyhf[FL]{Calculus formul\ae\ by Satya}
\fancyhf[FC]{2007/05/07}

%%%%%%%%%%%%%%%%%%%%%

\begin{document}

\thispagestyle{first}

\begin{center}
\LARGE { \textbf{Calculus} }
\author
\end{center}

\tableofcontents


\begin{enumerate}

\section{Limits}

\item
\begin{displaymath}
\lim_{x \to 0} \begin{array}{c}
\underline{\sin x} \\
x
\end{array}
=1
\qquad
\lim_{x \to 0} \cos x=1
\end{displaymath}

\item
\begin{displaymath}
\lim_{x \to 0}\begin{array}{c}
\underline{a^x-1} \\
x
\end{array} = \log a (a>0)
\qquad
\lim_{x \to 0}\begin{array}{c}
\underline{e^x-1} \\
x
\end{array} = \log e =1
\end{displaymath}


\item
$\frac{1}{n}=\alpha$, then $\alpha \to 0$ as $n \to \infty$
\begin{displaymath}
\lim_{n \to \infty}(1+
        \begin{array}{c}k \\ \overline{n}\end{array}) ^ n 
=
\lim_{\alpha \to 0}(1+
        k\alpha) ^ {1/\alpha}
=e^k
\end{displaymath}
If $k=1$,
\begin{displaymath}
\lim_{n \to \infty}(1+
        \begin{array}{c}1 \\ \overline{n}\end{array}) ^ n
=
\lim_{\alpha \to 0}(1+
        \alpha) ^ {1/\alpha} 
=e
\end{displaymath}

\item
\begin{displaymath}
\lim_{x \to 0}\begin{array}{c}
\underline{\log (1+x)} \\ x
\end{array} =1
\end{displaymath}

\item
\begin{displaymath}
\lim_{x \to a}\begin{array}{c}
\underline{x^n-a^n} \\ x-a
\end{array} =na^{n-1}
\end{displaymath}

\section{Derivatives}

\item
Provided the limit exists,
\begin{displaymath}
f'(x) = \lim_{h \to 0}\begin{array}{c}
\underline{f(x+h)-f(x)} \\ h
\end{array}
\end{displaymath}
\begin{displaymath}
f'(a) = \lim_{h \to 0}\begin{array}{c}
\underline{f(a+h)-f(a)} \\ h
\end{array}
\end{displaymath}

\item
$\derix{k}  =0$ where $k$ is a constant

\item
(In the following formul\ae, consider this one:)

If $y$ is a differentiable function of $u$ and $u$ is a differentiable function of $x$,
then $
\derix{y} = 
\derivx{y}{u} 
\cdot 
\derix{u}
$


\item
Remember, $
\derix{y} = 
\derivx{y}{u} 
\cdot 
\derix{u}
$


$
\begin{array}{|l|l|l|l|}
f(x) & \derx f(x) & f(u) (u=f(x)) &  \derx f(u) \\
\hline
x^n & nx^{n-1} & u^n & nu^{n-1}\derix{u} \\

\hline
\sin x & \cos x & \sin u & \cos u \derix{u} \\
\hline
\cos x & -\sin x & \cos u & -\sin u \derix{u} \\
\hline

\tan x & \sec^2 x & \tan u & \sec^2 u \derix{u} \\
\hline
\cot x & -\cosec^2 x & \cot u & -\cosec^2 u \derix{u} \\
\hline

\sec x & \sec x \tan x & \sec u & \sec u \tan u \derix{u} \\
\hline
\cosec x & - \cosec x \cot x & \cosec u & -\cosec u \cot u \derix{u} \\
\hline

e^x & e^x & e^u & e^u \derix{u} \\
\hline
a^x & a^x \ln a & a^u & a^u \ln a \derix{u} \\
\hline

\ln x & 1/x & \ln u & \frac{1}{u} \derix{u} \\
\hline

\end{array}
$

\item
$\derx{(u+v)} = \derix{u} + \derix{v}
\qquad
\derx{(u-v)} = \derix{u} - \derix{v}$


\item
$\derx{(uv)} = u\derix{v} + v\derix{u}$

\qquad
$\derx{}\begin{array}{c}
\underline{u} \\
{v}
\end{array} 
= 
\begin{array}{c}
\underline{ v\derix{u} - u \derix{v} } \\
v^2
\end{array}
v \ne 0$


\item
If $y=f(x)$ is a differentiable function of $x$ such that the inverse $x=g(y)$ is defined, then

$\derivx{x}{y} = \begin{array}{c}
1 \\
\overline{\derivx{y}{x}}
\end{array}
(\derivx{y}{x} \ne 0)$

\item
If $x=g(y)$ is a differentiable function of $y$ such that the inverse $y=f(x)$ is defined, then

$\derivx{y}{x} = \begin{array}{c}
1 \\
\overline{\derivx{x}{y}}
\end{array}
(\derivx{x}{y} \ne 0)$


\subsubsection{Derivatives of inverse trigonometric functions}

\end{enumerate}

\end{document}
