\documentclass[a4paper]{article}

\usepackage{verbatim}
\usepackage{fancyhdr}
\usepackage{latexsym}
\usepackage[colorlinks=true,
	linkcolor=blue,
	anchorcolor=black,
	citecolor=black,
	filecolor=black,
	menucolor=black,
	pagecolor=black,
	urlcolor=black
	]{hyperref}
\usepackage{tabularx}
\usepackage[left=1in,top=1in,right=1in]{geometry}

\pagestyle{fancy}
\fancyhf{}
\fancyhf[FR]{Page \thepage}
%\addtolength{\headheight}{3pt}
%\addtolength{\textheight}{48pt}
\renewcommand{\headrulewidth}{1pt}
\renewcommand{\footrulewidth}{1pt}
\fancypagestyle{first}{%
	\fancyhf[HR]{}
	}

\newcommand{\head}[1]{%
	\vspace{18pt}
	\large {\underline{#1}}
	\normalsize
	\vspace{12pt}
	}

\renewcommand{\author}{\small{\\by\\Satya\\https://www.github.com/satyap}}

\setlength{\parindent}{0pt}
\setlength{\parskip}{6pt}

%% the following newcommands are mostly for the math sheets
\newcommand{\elemof}[1]{%
    \in \mathbf{#1}
}
\newcommand{\degree}{^\circ}
\newcommand{\h}[2]{#2\footnote{\href{#1}{#1}} }
\newcommand{\p}{\vspace{12pt}}
\newcommand{\cosec}{\mathrm{cosec}}
\newcommand{\ud}{\mathrm{d}}
\newcommand{\half}{\frac{1}{2}}

%%%%%%%%%%%%%%%%%%%%%


\newcommand{\cplusd}{%
(\begin{array}{c}
\underline{C+D} \\
2
\end{array})
}
\newcommand{\cminusd}{%
(\begin{array}{c}
\underline{C-D} \\
2
\end{array})
}


\fancyhf[FL]{Math formul\ae\ by Satya}
\fancyhf[FC]{2007/10/08}

%%%%%%%%%%%%%%%%%%%%%

\begin{document}

\thispagestyle{first}

\begin{center}
\LARGE { \textbf{Mathematics, Algebra, and Geometry} }
\author
\end{center}

\tableofcontents

\section{Algebra}


\begin{enumerate}
\item $(a+b)^2 = a^2 + 2ab + b^2
\quad;\quad
a^2 + b^2 = (a+b)^2 - 2ab$

\item
$(a-b)^2 = a^2 - 2ab + b^2
\quad;\quad
a^2 + b^2 = (a-b)^2 + 2ab$

\item
$(a+b+c)^2 = a^2 +b^2+c^2 + 2(ab+bc+ca)$

\item
$(a+b)^3=a^3+b^3 + 3ab(a+b)
\qquad;\quad
a^3+b^3= (a+b)^3 - 3ab(a+b)$

\item
$(a-b)^3=a^3-b^3 - 3ab(a-b)
\qquad;\quad
a^3-b^3= (a-b)^3 + 3ab(a-b)$

\item
$a^2-b^2 = (a+b)(a-b)$

\item
$a^3-b^3 = (a-b)(a^2 + ab + b^2)$

\item
$a^3+b^3 = (a+b)(a^2 - ab + b^2)$

\item
$a^n-b^n = (a-b) ( a^{(n-1)} + a^{(n-2)}b + a^{(n-3)}+b^2 + \dots + b^{(n-1)}) $
$\quad \textrm{where } n \elemof{N}$

\item
$a^m.a^n = a^{m+n} \textrm{ where } m,n \elemof{Q}, a \elemof{R}$

\item
$\frac{a^m}{a^n}  = \left\{
\begin{array}{ll}
 a^{m-n} & \textrm{if } m>n \\
 1 & \textrm{if } m=n \\
 \frac{1}{a^{n-m}} & \textrm{if } m<n; m,n \elemof{Q}, a \elemof{R}, a \ne 0
\end{array}
\right.$

\item
$(a^m)^n = a^{mn} = (a^n)^m; m,n \elemof{Q}, a \elemof{R}$

\item
$(ab)^n = a^{n} \cdot b^n \textrm{ where } a,b \elemof{R}, n \elemof{Q}$

\item
$(\frac{a}{b})^n = \frac{a^n}{b^n} \textrm{ where } a,b \elemof{R}, n \elemof{Q}$

\item
$a^0 = 1 \textrm{ where } a \elemof{R}, a \neq 0$

\item
$a^{-n} = \frac{1}{a^n}, a^n = \frac{1}{a^{-n}} \textrm{ where }$
$a \elemof{R}, a \neq 0, n \elemof{Q}$

\item
$a^{p/q} = \sqrt[q]{a^p} \textrm { where } a \elemof{R}, a > 0, p,q \elemof{N}$

\item
$\textrm{If } a^m = a^n  \textrm{ where } a \elemof{R}, a \neq \pm 1, a \neq 0, 
\textrm{ then } m=n$

\item 
If $a^n=b^n$ where $ n \neq 0$, then  $a = \pm b$

\item
If $a,x,y \elemof{Q}$ and $\sqrt x,  \sqrt y$
are quadratic surds and if $a+\sqrt x = \sqrt y$,
then $a=0$ and $x=y$

\item 
If $a,b,x,y \elemof{Q}$ and $\sqrt x,  \sqrt y$
are quadratic surds and if $a+\sqrt x = b+\sqrt y$,
then $a=b$ and $x=y$

\subsection{Logarithms}

\item
$\log_a{mn} = log_a{m}+log_a{n}$
where $a,m,n$ are positive real numbers and $a \neq 1$,

\item
$\log_a(\frac{m}{n}) = log_a{m}-log_a{n}$
where $a,m,n$ are positive real numbers, $a \neq 1$,

\item
$\log_a m ^n = n \log_a m$
where $a$ and $m$ are positive real numbers, $a \neq 1, n \elemof{R}$

\item
$log_b a = 
%\frac{\log_k a}{\log_k b}
\begin{array}{c}\underline{\log_k a} \\ \log_k b \end{array}
$
where $a,b,k$ are positive real numbers, $b \neq 1, k \neq 1$

\item
$\log_b a = 
%\frac{1}{\log_a b}
\begin{array}{c} 1 \\ \overline{\log_a b} \end{array}
$ 
where $a,b$ are positive real numbers, $a \neq 1, b \neq 1$

\item
If $a,m,n$ are positive real numbers, $a \neq 1$, and if
$\log_a m = \log_a n$, then $m=n$

\subsection{Complex numbers}

\item
If $a+ib =0$ where $a,b \elemof{R}$ and $i = \sqrt{-1}$,
then $a=b=0$

\item
If $a+ib=x+iy$ where $a,b,x,y \elemof{R}$, $i = \sqrt{-1}$,
then $a=x$ and $b=y$

\subsection{Quadratic equations}

\item
The roots of the quadratic equation $ax^2 + bx+c=0; a \neq 0$
are 
%$\frac{-b \pm \sqrt{b^2-4ac}}{2a}$
$\begin{array}{c}
\underline{-b \pm \sqrt{b^2-4ac} } \\
2a
\end{array}$
The solution set of the equation is
$\Big\{ 
%\frac{-b + \sqrt \Delta}{2a}, \frac{-b - \sqrt \Delta}{2a} 
\begin{array}{ccc}
\underline{-b + \sqrt \Delta} &,& \underline{-b - \sqrt \Delta} \\
2a & & 2a \\
\end{array}
\Big\}$
where $\Delta =$ discriminant $=b^2-4ac$

\item
The roots are real and distinct if $\Delta > 0$

\item
The roots are real and coincident if $\Delta = 0$

\item
The roots are non-real if $\Delta < 0$

\item
If $\alpha$ and $\beta$ are the roots of the equation 
$ax^2 + bx+c=0; a \neq 0$,
then
\begin{enumerate}
\item
$\alpha +\beta = -\frac{b}{a} = - \frac{\mathrm{Coeff. of }x}{\mathrm{Coeff. of }x^2}$
\item
$\alpha \beta = \frac{c}{a} = \frac{\mathrm{Const. term}}{\mathrm{Coeff. of }x^2}$
\end{enumerate}

\item
The quadratic equations whose roots are $\alpha$ and $\beta$
is $(x-\alpha)(x-\beta)=0$
i.e. $x^2 - (\alpha + \beta)x + \alpha\beta =0$
i.e. $x^2 -Sx + P=0$
where $S$ = sum of the roots and $P$ = product of the roots.

\subsection{Sequences and series}

\item 
For an Arithmetic Progression (A.P.) whose first term is `$a$' and common
difference is `$d$',
\begin{enumerate}
\item $n^{th}$ term $= t_n = a + (n-1) d$
\item The sum of the first $n$ terms = 
$S_n = \frac{n}{2}(a+l) = \frac{n}{2} \{2a+(n-l)d\}$
where $l$ = last term = $a+(n-1)d$
\end{enumerate}

\item
For a Geometric Progression (G.P.) whose first term is `$a$' and common
ratio is `$r$',
\begin{enumerate}
\item $n^{th}$ term $= t_n = ar^{n-1}$
\item The sum of the first n terms = $S_n  
\begin{array}{lcl}
=& \underline{a(1-r^n)} & $ if $r \le 1 \\
 & 1-r &  \\
=& \underline{a(r^n-1)} & $ if $r \ge 1 \\
 & r-1      &  \\
=& na       & $ if $ r=1 \\
\end{array}$
\end{enumerate}

\item
For any sequence $\{ t_n \}, S_n - S_{n-1} = t_n$ where
$S_n=$ sum of the first $n$ terms.

\item
$\sum_{r=1}^n r = 1+2+3+4+ \dots +n = \frac{n}{2}(n+1)$

\item
$\sum_{r=1}^n r^2 = 1^2+2^2+3^2+4^2+ \dots +n^2 = \frac{n}{6}(n+1)(2n+1)$

\item
$\sum_{r=1}^n r^3 = 1^3+2^3+3^3+4^3+ \dots +n^3 = \frac{n^2}{4}(n+1)^2$


\subsection{Factorials and Probability}

\item
$\lfloor n = n! = 1.2.3.4. \dots (n-1)n$

\item
$n!=n(n-1)! = n(n-1)(n-2)! = n(n-1)(n-2)(n-3)! = \dots$

\item
$0! = 1$

\item
${}^nP_r = \frac{n!}{(n-r)!} $
$\qquad$
${}^nC_r = \frac{n!}{r!(n-r)!}$

\item
${}^nP_0=1 \qquad {}^nP_1=n \qquad {}^nP_2=n(n-1) \qquad {}^nP_3=n(n-1)(n-2)\dots$

\item
${}^nC_0=1 \qquad {}^nC_1=n \qquad {}^nC_2=\frac{n(n-1)}{2!}
 \qquad {}^nC_3=\frac{n(n-1)(n-2)}{3!}\dots$

\item
$\begin{array}{c}
{}^nP_r \\
\hline
{}^nC_r \\
\end{array}
= r!$

\item
${}^nC_r = {}^nC_{n-r} \qquad
{}^nC_{r-1} +{}^nC_{r} = {}^{n+1}C_{r}$

\item
If ${}^nC_x = {}^nC_y$, then $x=y$ or $x+y=n$

\item
If $a,b \elemof{R}$ and $n \elemof{N}$, then
$(a+b)^n = {}^nC_0 a^n b^0 + {}^nC_1 a^{n-1} b^1 + {}^nC_2 a^{n-2} b^2 +
\dots + {}^nC_r a^{n-r} b^r + \dots
{}^nC_n a^0 b^n
$

\item
The general term in the expansion of $(a+b)^n$ is given by
$t_{r+1} = {}^nC_r a^{n-r} b^r$

\section{Trigonometry}

\item
$\pi^c = 180 \degree;
\qquad
1^c = (\begin{array}{c}180 \degree\\ \hline\pi\end{array}) 
= 57 \degree 17'44.8";
\qquad
1 \degree = 0.0175 radians
$

\item
$s=r\theta$ where 
$s=$ arc length, $r=$ radius and $\theta$ 
is the angle in radians subtended by the arc at the centre of the circle.

\item
$A=\frac{1}{2} r^2 \theta = \frac{1}{2}rs$ where $A=$ area
of sector of a circle,
$s=$ arc length, $r=$ radius and $\theta$ 
is the angle in radians subtended by the arc at the centre of the circle.

\section{Trigonometric identities}

\item
\begin{equation}\label{eq:sincos}
\sin^2 \theta + \cos^2 \theta = 1
\end{equation}

\item
These follow by dividing \ref{eq:sincos} by $\cos^2\theta$ and $\sin^2 \theta$
respectively:
\begin{eqnarray}
\sec^2 \theta &=& 1+ \tan^2 \theta \\
\cosec^2 \theta &=& 1+ \cot^2 \theta
\end{eqnarray}

\item
These follow because $\sec$, $\cosec$, $\cot$ are reciprocals of 
$\cos$, $\sin$, and $\tan$:
\begin{eqnarray}
\cos\theta\sec\theta&=&1 \\
\sin\theta\cosec\theta&=&1 \\
\tan\theta\cot\theta&=&1
\end{eqnarray}

\item
$\tan \theta = 
\begin{array}{c} \sin \theta \\ \hline \cos\theta \end{array}
\qquad
\cot \theta =
\begin{array}{c} \cos \theta \\ \hline \sin\theta\end{array}
$

\subsection{Angular relations}

\item
$-1 \leq \cos \theta \leq 1$ i.e. $| \cos \theta | \leq 1$
\qquad
$-1 \leq \sin \theta \leq 1$ i.e. $| \sin \theta | \leq 1$

\item
\begin{eqnarray}
\sin n \pi = & 0, & n\elemof{I} \\
\cos n \pi = & (-1)^n, & n\elemof{I} \\
\sin (2n+1)\frac{\pi}{2} = & (-1)^n, & n \elemof{I} \\
\cos (2n+1)\frac{\pi}{2} = & 0, & n \elemof{I}
\end{eqnarray}

\item
$\cos (-\theta) = \cos \theta
\qquad
\sin (-\theta) = -\sin \theta
\qquad
\tan (-\theta) = -\tan \theta$

\item
$\begin{array}{|c|c|c|c|c|c|}
\hline
 & 0\degree & 30\degree & 45 \degree & 60 \degree & 90 \degree \\
\hline
\sin & 0 & \half & \frac{1}{\sqrt{2}} & \frac{\sqrt 3}{2} & 1 \\
\hline
\cos & 1 & \frac{\sqrt 3}{2} & \frac{1}{\sqrt{2}} & \half & 1 \\
\hline
\tan & 0 & \frac{1}{\sqrt 3} & 1 & \sqrt 3 & \infty \\
\hline
 & 0 & \frac{\pi}{6} & \frac{\pi}{4} & \frac{\pi}{3} & \frac{\pi}{2}\\
\hline
\end{array}$


\item
$\sin (n\pi \pm \theta) = \textbf{(sign)} \sin \theta, n \elemof{I}$, sign is determined by quadrant to which the angle belongs.
\item
$\cos (n\pi \pm \theta) = \textbf{(sign)} \cos \theta, n \elemof{I}$


\item
$\sin ((2n+1)\frac{\pi}{2} \pm \theta) = \textbf{(sign)} \cos \theta, n \elemof{I}$, sign is determined by quadrant to which the angle belongs.
\item
$\cos ((2n+1)\frac{\pi}{2} \pm \theta) = \textbf{(sign)} \sin \theta, n \elemof{I}$

\item
If $\sin \theta = \sin \alpha$, then $\theta=n\pi + (-1)^n \alpha, n \elemof{I}$
\item
If $\cos \theta = \cos \alpha$, then $\theta=2n\pi + \alpha, n \elemof{I}$
\item
If $\tan \theta = \tan \alpha$, then $\theta=n\pi + \alpha, n \elemof{I}$

\subsection{Sine-Cosine-Tangent Combos}

\item
$\begin{array}{rcl}
\sin(\alpha + \beta) &=& \sin\alpha \cos\beta + \cos\alpha \sin\beta \\
\sin(\alpha - \beta) &=& \sin\alpha \cos\beta - \cos\alpha \sin\beta \\
\cos(\alpha + \beta) &=& \cos\alpha \cos\beta - \sin\alpha \sin\beta \\
\cos(\alpha - \beta) &=& \cos\alpha \cos\beta + \sin\alpha \sin\beta
\end{array}$
%$\begin{array}{c|cc}
% & \alpha + \beta & \alpha - \beta \\
%\hline
%\sin & sc+cs & sc-cs \\
%\cos & cc-ss & cc+ss 
%\end{array}$

\item
$\begin{array}{rcl}
2\sin\alpha \cos\beta &=& \sin (\alpha+\beta) + \sin (\alpha-\beta) \\
2\cos\alpha \sin\beta &=& \sin (\alpha+\beta) - \sin (\alpha-\beta) \\
2\cos\alpha \cos\beta &=& \cos (\alpha+\beta) + \cos (\alpha-\beta) \\
2\sin\alpha \sin\beta &=& \cos (\alpha-\beta) - \cos (\alpha+\beta) \\
\end{array}$

\item
$\begin{array}{rcl}
\sin C + \sin D &=& 2\sin\cplusd \cos\cminusd \\
\sin C - \sin D &=& 2\cos\cplusd \sin\cminusd \\
\cos C + \cos D &=& 2\cos\cplusd \cos\cminusd \\
\cos C - \cos D &=& -2\sin\cplusd \sin\cminusd
\end{array}$

\item
Tan relations:
\begin{equation}\label{eq:tanaplusb}
\tan(\alpha + \beta) =
\begin{array}{c}
\tan\alpha + \tan\beta \\
\overline{1 - \tan\alpha \tan\beta}
\end{array}
\quad
\tan(\alpha - \beta) =
\begin{array}{c}
\tan\alpha - \tan\beta \\
\overline{1 + \tan\alpha \tan\beta}
\end{array}
\end{equation}

\subsection{Halving angles}

(In the following, sometimes we use $2\theta = \alpha$, and $\theta=\alpha/2$)

\item
$\sin 2\theta = 2\sin\theta \cos\theta$

\item
\begin{equation}\label{eq:cos2theta}
\cos 2\theta = \cos^2\theta - \sin^2 \theta = 2\cos^2 \theta - 1 = 1 - 2\sin^2\theta
\end{equation}

\item
By \ref{eq:cos2theta},
$\cos^2 \theta = \half (1+ \cos 2\theta)$
and
$\sin^2 \theta = \half (1- \cos 2\theta)$

\item
By \ref{eq:tanaplusb},
$\tan 2 \theta = 
\begin{array}{c}
2 \tan \theta \\
\overline{  1-  \tan^2 \theta}
\end{array}
$

\item
If $\tan \theta = t$, then
$\cos 2\theta = 
\begin{array}{c}
1-t^2 \\
\overline{1+t^2}
\end{array}$
\qquad
$\sin 2\theta = 
\begin{array}{c}
2t \\
\overline{1+t^2}
\end{array}$

\subsubsection{More angular trigonometric relations}

\item
$\sin 3\theta = 3\sin\theta - 4 \sin^3\theta$

\item
$\cos 3\theta = 4\cos^3 \theta - 3\cos \theta$

\item
$\tan 3 \theta = \begin{array}{c}
\underline{3\tan\theta - \tan ^3\theta} \\
1-3\tan^2\theta
\end{array}$

\subsection{Trigonometric Rules}

\item
Sine Rule: In a $\Delta ABC$,
$\begin{array}{ccccccc}
a & = & b & = & c & = & 2R \\
\overline{\sin A} & & \overline{\sin B} & & \overline{\sin C} & &
\end{array}$
where $R$ is the circum-radius of the triangle.

\item
Cosine Rule: In a $\Delta ABC$,
$\begin{array}{rcl}
a^2 &=& b^2 + c^2 -2bc\cos A \\
b^2 &=& c^2 + a^2 -2ca\cos B \\
c^2 &=& a^2 + b^2 -2ab\cos C
\end{array}$

\item
Projection Rule: In a $\Delta ABC$,
$\begin{array}{rcl}
a &=& b\cos C + c\cos B \\
b &=& c\cos A + a\cos C \\
c &=& a\cos B + b\cos A
\end{array}$

\item
Area of $\Delta ABC = 
\half bc \sin A =
\half ca \sin B =
\half ab \sin C$
(half of the product of the length of two sides and the sine of the angle between them.)

\subsection{Inverse trigonometric relations}

\item
$\cosec^{-1}\frac{1}{x} = \sin ^{-1} x$
\qquad
$\sec^{-1}\frac{1}{x} = \cos ^{-1} x$
\qquad
$\cot^{-1}\frac{1}{x} = \tan ^{-1} x$

\item
$\sin^{-1}(\sin x) = x$ for $-\frac{\pi}{2} \le x \le \frac{\pi}{2}$

$\cos^{-1}(\cos x) = x$ for $0 \le x \le \pi$

$\tan^{-1}(\tan x) = x$ for $-\frac{\pi}{2} < x < \frac{\pi}{2}$

\item
$\sin^{-1}(-x) = -\sin^{-1}x$ for $-1 \le x \le 1$

$\cos^{-1}(-x) = \pi - \cos^{-1}x$ for $-1 \le x \le 1$

$\tan^{-1}(-x) = -\tan^{-1}x$ $\forall x \elemof{R}$


\item
If $x>0, y>0$

$\tan^{-1}x - \tan^{-1}y = \tan^{-1}(
\begin{array}{c}
x-y \\
\overline{1+xy}
\end{array}
)$

Additionally, if $xy<1$, then 
$\tan^{-1}x + \tan^{-1}y = \tan^{-1}(
\begin{array}{c}
x+y \\
\overline{1-xy}
\end{array}
)$

and, if $xy>1$, then
$\tan^{-1}x + \tan^{-1}y = \pi + \tan^{-1}(
\begin{array}{c}
x+y \\
\overline{1-xy}
\end{array}
)$


\section{Coordinate geometry}

\item
\textbf{Distance formula:}
Distance between two points $P(x_1, y_1)$ and $Q(x_2,y_2)$ is given by

$d(P,Q) = \sqrt{(x_2-x_1)^2 + (y_2-y_1)^2}$ 

Distance of the point $P(x,y)$ from the origin is

$d(O,P) = \sqrt{x^2+y^2}$

\item
\textbf{Section formula:}
If $A=(x_1,y_1)$, $B=(x_2,y_2)$ and $C(x,y)$ 
divides $AB$ in the ratio $m:n$ then

$\begin{array}{rclrc}
x= & \underline{mx_2+nx_1}  &, \qquad & y=& \underline{my_2+ny_1} \\
 & m+n & & & m+n 
\end{array}$

\item
\textbf{Mid-point formula:}
If $A=(x_1,y_1)$, $B=(x_2,y_2)$ and $C(x,y)$ 
is the midpoint of $AB$ then

$\begin{array}{rclrc}
x= & \underline{x_1+x_2}  &, \qquad & y=& \underline{y_1+y_2} \\
 & 2 & & & 2 
\end{array}$

\subsection{Triangles}

\item
\textbf{Centroid formula:}
If $G(x,y)$ is the centroid of a triangle whose vertices

are $A(x_1,y_1), B(x_2,y_2), C(x_3,y_3)$, then

$\begin{array}{rclrc}
x= & \underline{x_1+x_2+x_3}  &, \qquad & y=& \underline{y_1+y_2+y_3} \\
 & 3 & & & 3 
\end{array}$

\item
Area of a triangle whose vertices are $A(x_1,y_1), B(x_2,y_2), C(x_3,y_3)$ is

$\Delta = \half|
x_1(y_2-y_3)+
x_2(y_3-y_1)+
x_3(y_1-y_2)|$

Area of a triangle whose vertices are $O(0,0), A(x_1,y_1), B(x_2,y_2)$ is

$\Delta = \half|
x_1y_2
-x_2y_1|$

\subsection{Straight lines}

\item
Equation of the x-axis is $y=0$

Equation of the y-axis is $x=0$

\item
Equation of straight line parallel to x-axis and passing through point $P(a,b)$ is $y=b$

Equation of straight line parallel to y-axis and passing through point $P(a,b)$ is $x=a$

\item
Slope of a straight line

$m = \tan \theta = 
\begin{array}{c}
\underline{y_2-y_1} \\
x_2-x_1
\end{array}
$ 

where $\theta$ is the inclination of the straight line and $(x_1,y_1)$ and $(x_2,y_2)$ are any two points on the line.

\item
Equation of a straight line

\begin{tabular}{lc}
Slope-origin form & $y=mx$ \\
Point-origin form & $xy_1 = yx_1$ \\
Slope-intercept form & $y=mx+c$ \\
Point-slope form & $y-y_1 = m(x-x_1)$ \\
Two-points form & $
\begin{array}{ccc}
y-y_1 & = & x-x_1 \\
\overline{y_2-y_1} & &  \overline{x_2-x_1} \\
\end{array}
$ \\
Double-intercept form & $xb + ya = ab$ \\
Normal form & $x\cos\alpha + y\sin\alpha = p$ \\
\end{tabular}

\item
Parametric equations of a straight line are

$x=x_1 + r\cos\theta
\qquad
y=y_1 + r\sin\theta$

\item
General equation of a straight line is $ax+by+c=0$

For this line, 

Slope=$-\begin{array}{c}\underline{a} \\ b\end{array}$
\qquad
x-intercept = $-\begin{array}{c}\underline{c} \\ a\end{array}$
\qquad
y-intercept = $-\begin{array}{c}\underline{c} \\ b\end{array}$

\subsection{Line pairs}

\item
The acute angle between
two straight lines with slopes $m$ and $m'$
is $\tan \theta=\Big|
\begin{array}{c}
m-m' \\
\overline{1+mm'}
\end{array}\Big|$

\item
The straight lines with slopes $m$ and $m'$
are mutually perpendicular iff $mm'=-1$

\item
The straight lines with slopes $m$ and $m'$
are parallel to each other iff $m=m'$

\item
Any line parallel to the line $ax+by+c=0$ has an equation of the form 
$ax+by+k=0$ 
where $k \elemof{R}$

\item
Any line perpendicular to the line $ax+by+c=0$ has an equation of the form 
$bx-ay+k=0$
where $k \elemof{R}$

\item
The acute angle between the two 
straight lines $ax+by+c=0$, $a'x+b'y+c'=0$ 
is given by
\begin{equation}\label{eq:slopebetweenlines}
\tan\theta=\Big|\begin{array}{c}
ab'-a'b \\
\overline{aa'+bb'}
\end{array}\Big|
\end{equation}

\item
By \ref{eq:slopebetweenlines}
The straight lines $ax+by+c=0$, $a'x+b'y+c'=0$ are

mutually perpendicular if $aa'+bb'=0$

parallel if $ab'=a'b$

identical if $\begin{array}{ccccc}
a & = & b & = & c \\
\overline{a'} & & \overline{b'} & & \overline{c'}
\end{array}$

\item
The perpendicular distance of the point $P(x_1,y_1)$ from the straight line
$ax+by+c=0$ is

$\Big|
\begin{array}{c}
\underline{ax_1+by_1+c} \\
\sqrt{a^2+b^2}
\end{array}
\Big|$


The perpendicular distance of the origin ($x_1=0, y_1=0$) from the straight line
is

$\Big|
\begin{array}{c}
c \\
\overline{ \sqrt{a^2+b^2} }
\end{array}
\Big|$

The distance between two parallel straight lines
$ax+by+c=0$ and
$ax+by+c'=0$ is

$\Big|
\begin{array}{c}
c-c' \\
\overline{ \sqrt{a^2+b^2} }
\end{array}
\Big|$

\section{Mensuration}

\subsection{Solids}

\item
Sphere of radius $r$

Volume = $\frac{4}{3}\pi r^3$

Surface area = $4\pi r^2$

\item
Right circular cone, radius $r$ height $h$ slant height $l$

Volume = $\frac{1}{3} \pi r^2 h$

Curved surface area=$\pi rl$

Total surface area = $\pi rl + \pi r^2$

\item
Right circular cylinder, radius $r$ height $h$

Volume = $\pi r^2 h$

Curved surface area = $2\pi rh$

Total surface area = $2\pi rh + 2\pi r^2$

\item
Cube with side length $x$

Volume=$x^3$

Surface area = $6x^2$

\item
Volume of a rectangular parallelopiped = length x breadth x height

\subsection{Plane figures}

\item
Circle of radius $r$

Area = $\pi r^2$

Perimeter = $2\pi r$

\item
Triangle, area = $\half$ x base x height

\item
Rectangle, area = length x breadth, perimeter = 2 x (length + breadth)

\item
Square, area = (side)$^2$, perimeter = 4 x side

\item
Area of a trapezium = $\half$ x (sum of parallel sides) x (distance between the parallel sides)

\item
Area of an equilateral triangle = 
$\begin{array}{c}
\underline{\sqrt 3} \\
4 \\
\end{array}
a^2=
\begin{array}{c}
1 \\
\overline{\sqrt 3} \\
\end{array} p^2$

where $a$ is the length of a side and $p$ is the length of an altitude.

\section{Numerical Methods}

\item Bisection Method:
$c=\begin{array}{c}
\underline{x_1+x_2} \\
2\end{array}
$

\item False Position (Regula-Falsi) Method:
$\begin{array}{|ccc|cc}
x_n+1 & 0      & 1 & & \\
a     & f(a)   & 1 & = & 0 \\
x_n   & f(x_n) & 1 & &
\end{array}$

\item Newton-Raphson Method:
$x_{n+1} = x_n -
\begin{array}{c}
f(x_n) \\
\overline{f'(x_n)}
\end{array}$

$f(x) = $ $n^{th}$ degree polynomial $<=> \Delta^nf(x)$ constant

$y_{n+1}=(1+\Delta)y_n \qquad y_i=(1-\nabla)y_{i+1}$

$1+\Delta=E \qquad E^{-1}=1-\nabla$

\item Forward interpolation (Newton-Gregory):
$u=
\begin{array}{c}
\underline{\overline{x} - x_0} \\
n
\end{array}$

$f(\overline{x}) = f(x_0) + u \Delta f(x_0) +
\begin{array}{c}
\underline{u(u-1)} \\
2!
\end{array}
\Delta^2f(x_0) + \dots$

\item Backward interpolation:
$\nu=
\begin{array}{c}
\underline{\overline{x} - x_n} \\
n
\end{array}$

$f(\overline{x}) = f(x_n) + \nu \nabla f(x_n) +
\begin{array}{c}
\underline{\nu(\nu+1)} \\
2!
\end{array}
\nabla^2f(x_n) + \dots$

\item Trapezoidal Rule:
$\int_a^b f(x) \ud x = h(
\begin{array}{c}
\underline{y_0+y_n} \\
2
\end{array}
+y_1+y_2+\dots+y_{n-1})$

\item Simpson's (one-third) rule:
$\int_a^b y \ud x=\frac{1}{3} h [(y_0 + y_n) + 4(y_1 + y_3 + \dots + y_{n-1})
+2(y_2 + y_4 + \dots + y_{n-2})]$
$n$ is even.


\end{enumerate}

\end{document}
