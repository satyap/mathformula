\documentclass[a4paper]{article}

\usepackage{verbatim}
\usepackage{fancyhdr}
\usepackage{latexsym}
\usepackage[colorlinks=true,
	linkcolor=blue,
	anchorcolor=black,
	citecolor=black,
	filecolor=black,
	menucolor=black,
	pagecolor=black,
	urlcolor=black
	]{hyperref}
\usepackage{tabularx}
\usepackage[left=1in,top=1in,right=1in]{geometry}

\pagestyle{fancy}
\fancyhf{}
\fancyhf[FR]{Page \thepage}
%\addtolength{\headheight}{3pt}
%\addtolength{\textheight}{48pt}
\renewcommand{\headrulewidth}{1pt}
\renewcommand{\footrulewidth}{1pt}
\fancypagestyle{first}{%
	\fancyhf[HR]{}
	}

\newcommand{\head}[1]{%
	\vspace{18pt}
	\large {\underline{#1}}
	\normalsize
	\vspace{12pt}
	}

\renewcommand{\author}{\small{\\by\\Satya\\https://www.github.com/satyap}}

\setlength{\parindent}{0pt}
\setlength{\parskip}{6pt}

%% the following newcommands are mostly for the math sheets
\newcommand{\elemof}[1]{%
    \in \mathbf{#1}
}
\newcommand{\degree}{^\circ}
\newcommand{\h}[2]{#2\footnote{\href{#1}{#1}} }
\newcommand{\p}{\vspace{12pt}}
\newcommand{\cosec}{\mathrm{cosec}}
\newcommand{\ud}{\mathrm{d}}
\newcommand{\half}{\frac{1}{2}}

%%%%%%%%%%%%%%%%%%%%%


\fancyhf[FL]{EMF formul\ae\ by Satya}

%%%%%%%%%%%%%%%%%%%%%

\begin{document}

\thispagestyle{first}

\begin{center}
\LARGE { \textbf{Electromagnetic Fields \& Waves} }
\author
\end{center}

\tableofcontents

\section{Div, Grad, Curl, Laplacian}

{\large Divergence:}
\begin{displaymath}
\textrm{Cartesian: }
\nabla \cdot \mathbf D =
\frac{\partial D_x}{\partial x} +
\frac{\partial D_y}{\partial y} +
\frac{\partial D_z}{\partial z} 
\end{displaymath}

\begin{displaymath}
\textrm{Cylindrical: }
\nabla \cdot \mathbf D =
\frac 1 r \frac \partial {\partial r} (r D_X)+
\frac 1 r \frac {\partial D_\theta} {\partial \theta}+
\frac{\partial D_z}{\partial z}
\end{displaymath}

\begin{displaymath}
\textrm{Spherical: }
\nabla \cdot \mathbf D =
\frac 1 {r^2} \frac \partial {\partial r} (r^2 D_X)+
\frac 1 {r \sin \theta} \frac \partial {\partial\theta} (D_\theta \sin\theta) +
\frac 1 {r \sin\theta} \frac{\partial D_\phi}{\partial\phi}
\end{displaymath}

{\large Gradient:}
\begin{displaymath}
\textrm{Cartesian: }
\nabla V =
\frac {\partial V}{\partial x} \vec{i}+
\frac {\partial V}{\partial y} \vec{j}+
\frac {\partial V}{\partial z} \vec{k}
\end{displaymath}


\begin{displaymath}
\textrm{Cylindrical: }
\nabla V =
\frac{\partial V}{\partial r} a_r +
\frac 1 r \frac {\partial V}{\partial \theta} a_\theta +
\frac {\partial V}{\partial z} a_z
\end{displaymath}

\begin{displaymath}
\textrm{Spherical: }
\nabla V=
\frac{\partial V}{\partial r}a_r +
\frac 1 r \frac {\partial V}{\partial \theta} a_\theta +
\frac 1 {r \sin \theta} \frac {\partial V}{\partial \phi} a_\phi
\end{displaymath}


{\large Curl:}

\begin{displaymath}
\textrm{Cartesian: }
\nabla \times \mathbf H =
(\frac{\partial H_z}{\partial y} - \frac{\partial H_y}{\partial z}) a_x +
(\frac{\partial H_x}{\partial z} - \frac{\partial H_z}{\partial x}) a_y +
(\frac{\partial H_y}{\partial x} - \frac{\partial H_x}{\partial y}) a_z
\end{displaymath}

\begin{displaymath}
\textrm{Cylindrical: }
\nabla \times \mathbf H =
(\frac 1 r \frac{\partial H_z}{\partial \theta} - \frac{\partial
H_\theta}{\partial z}) a_r +
(\frac{\partial H_r}{\partial z} - \frac{\partial H_z}{\partial
r}) a_\theta +
\frac 1 r [\frac{\partial (r H_\theta)}{\partial r} -
\frac{\partial H_r}{\partial \theta}] a_z
\end{displaymath}

Spherical:

\begin{eqnarray}
\nabla \times \mathbf H = 
\frac 1 {r\sin \theta} [ \frac{\partial H_z}{\partial \theta} -
\frac{\partial
H_\theta}{\partial \phi}) a_r + 
\frac 1 r [
\frac 1 {\sin\theta}
\frac{\partial H_r}{\partial \phi} - \frac{\partial(rH_\phi)}{\partial
r}) a_\theta +
{}\nonumber\\
\frac 1 r [\frac{\partial (r H_\theta)}{\partial r} -
\frac{\partial H_r}{\partial \theta}] a_\phi
\nonumber
\end{eqnarray}

{\large Laplacian:}

\begin{displaymath}
\textrm{Cartesian: }
\nabla^2 V = 
\frac{\partial ^2 V}{\partial x^2} +
\frac{\partial ^2 V}{\partial y^2} +
\frac{\partial ^2 V}{\partial z^2} 
\end{displaymath}

\begin{displaymath}
\textrm{Cylindrical: }
\nabla^2 V = 
\frac 1 r
\frac \partial {\partial r}
(r \frac{\partial V}{\partial r} ) +
\frac 1 {r^2}
\frac {\partial^2V} {\partial \theta^2} +
\frac{\partial ^2 V}{\partial z^2} 
\end{displaymath}

\begin{displaymath}
\textrm{Spherical: }
\nabla ^2 V =
\frac 1 {r^2}
\frac \partial {\partial r}
(r^2 \frac{\partial V}{\partial r} )+
\frac 1 {r^2 \sin\theta}
\frac \partial {\partial \theta}
( \sin\theta \frac{\partial V}{\partial \theta}) +
\frac 1 {r^2 \sin\theta}
\frac{\partial^2 V}{\partial\phi^2} 
\end{displaymath}

\section{Vector identities}

\begin{displaymath}
\nabla\cdot(\nabla\times\mathbf F ) = 0
\qquad
\nabla \times(\nabla f)=0
\qquad
\nabla \cdot (\nabla f)=\nabla^2f
\qquad
\end{displaymath}

\begin{displaymath}
\nabla \times(\nabla \times \mathbf F) =
\nabla (\nabla \cdot \mathbf F) - 
\nabla^2 f
\end{displaymath}

\begin{displaymath}
\nabla(fg)=f\nabla g + g\nabla f
\qquad 
\nabla \cdot(f \mathbf G) =
\nabla f \cdot \mathbf G + f \nabla \cdot \mathbf G
\end{displaymath}

\begin{displaymath}
\nabla \times (f\mathbf G) =
\nabla f \times \mathbf G + f \times \nabla \mathbf G
\end{displaymath}

\begin{displaymath}
\nabla(\mathbf F \cdot \mathbf G) =
(\mathbf F \cdot \nabla) \mathbf G +
(\mathbf G \cdot \nabla) \mathbf F +
\mathbf F \times (\nabla \times \mathbf G) +
\mathbf G \times (\nabla \times \mathbf F)
\end{displaymath}

\begin{displaymath}
\nabla \cdot )\mathbf F \times \mathbf G) =
\mathbf G \cdot (\nabla \times \mathbf F) -
\mathbf F \cdot (\nabla \times \mathbf G)
\end{displaymath}

\begin{displaymath}
\nabla \times (\mathbf F \times \mathbf G) =
\mathbf F (\nabla \cdot \mathbf G) -
\mathbf G (\nabla \cdot \mathbf F) +
(\mathbf G \cdot \nabla) \mathbf F -
(\mathbf F \cdot \nabla) \mathbf G
\end{displaymath}


\section{Theorems}

{\large Divergence theorem:}
\begin{displaymath}
\int_v \nabla \cdot \mathbf A \  dv= \int_s \mathbf A \cdot ds
\end{displaymath}

{\large Stoke's theorem:}
\begin{displaymath}
\int_s \nabla \times \mathbf A \ ds = \int_c \mathbf A \cdot dl
\end{displaymath}


{\large Helmholtz' theorem: }\\
A vector field is completely specified by its divergence and
curl. Conversely, any vector field may be expressed as the sum of an
irrotational vector and a solenoidal vector.

\begin{displaymath}
\textrm{Potential }\phi = -
\int_\infty^P \mathbf E \cdot \mathbf{dl} \quad \textrm{(volts)}
\end{displaymath}

Spherical conducting shell: 
\begin{displaymath}
{}\nonumber\\
\quad 
\textrm{Inside: }
\quad
E=0
\quad
\phi=\frac Q {4 \pi \epsilon_0 R}
\end{displaymath}

\begin{displaymath}
\textrm{Outside: }
E= \frac Q {4 \pi \epsilon_0 r^2}
\quad
\mathbf \phi= \frac Q {4 \pi \epsilon_0 r}
\end{displaymath}
(small $r$ is outer radius of shell)


Charged wire (cylinder) of infinite length:
\begin{displaymath}
\phi_a-\phi_b=\frac{Q_l}{2\pi\epsilon_0} \ln \frac{r_b}{r_a}
\end{displaymath}

{\large Gauss' Law:}
\begin{displaymath}
\oint_s \mathbf {D\cdot ds} = \int_v Q_v dv
\quad
\textrm{ or }
\quad
\nabla\cdot\mathbf D=Q_v
\end{displaymath}

\begin{displaymath}
\textrm{Laplace: }
\nabla^2\phi=0
\qquad
\textrm{Poisson: }
\nabla^2\phi=-\frac{Q_v}{\epsilon}
\end{displaymath}


\end{document}
